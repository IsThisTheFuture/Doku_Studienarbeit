%!TEX root = ../dokumentation.tex

\chapter{Theoretische Grundlagen}
\section{Elektromagnetische Wellen}
Als elektromagnetische Welle werden räumlich ausbreitende Wellen bezeichnet, die aus elektrischen und magnetischen Feldern besteht. Die laufende Welle breitet sich entlang der Ausbreitungsgeschwindigkeit mit einer Geschwindigkeit von \[c= 3*10^8 m/s\] aus. Die elektrischen und magnetischen Feldvektoren der Welle stehen orthogonal aufeinander, wie in der Abbildung \ref{elektromagnetische Welle} gut zu erkennen.

\begin{figure}[ht]
	\centering
	\includegraphics[width=0.75\textwidth]{em-welle.png}
	\caption[Elektromagnetische Welle mit senkrecht
	aufeinander stehendem elektrischem und magnetischem Feld]{Elektromagnetische Welle mit senkrecht aufeinander stehendem elektrischem und magnetischem Feld. Quelle: \cite[Harten, S. 130]{Harten:2017}} 
	\label{elektromagnetische Welle}
\end{figure}

Der Frequenzbereich, auch Spektrum einer solchen elektromagnetischen Welle reicht von langsamen Radiowellen, Infrarotwellen über den Bereich des sichtbaren Lichts bis hin zur Röntgenstrahlung und der extrem kurzwelligen Gammastrahlung. In der Abbildung \ref{frequenzbereiche} werden die Frequenzbereiche detaillierter dargestellt.

Die elektromagnetischen Wellen, die auf unsere Erde einwirken werden auch natürliche Strahlung genannt. Sie ermöglicht das Leben auf der Erde, da die Energiezufuhr des Lebens auf der Erde über infrarot Wellen der Sonne gegeben ist.
Elektromagnetische Wellen benötigen kein Medium, um sich auszubreiten. Sie können sich daher auch über weiteste Entfernungen ausbreiten. Sie bewegen sich im Vakuum unabhängig von ihrer Frequenz mit Lichtgeschwindigkeit fort. Elektromagnetische Wellen können sich aber auch in Materie ausbreiten wie etwa Gas oder Flüssigkeit, jedoch verringert sich dabei ihre Geschwindigkeit.

Die Existenz der elektromagnetischen Wellen folgt aus den Maxwell-Gleichungen. Sie wurden 1886 von H.Hertz erstmals durch den elektrischen Schwingkreis erzeugt. Diese werden im folgenden Kapitel genauer erklärt.

\section{Maxwellsche Gleichungen}

Die Maxwell-Gleichungen sind grundlegende Gleichungen der Elektrodynamik. Sie dienen zur Beschreibung der Phänomene des Elektromagnetismus und sie sind damit ein wichtiger Teil der Elektrodynamik. Mit Hilfe der Gleichungen können alle elektrischen und magnetischen Effekte beschrieben werden, wie den Zusammenhang elektrischer und magnetischer Felder untereinander. Außerdem die Korrelation von elektrischer Ladungen und elektrischen Strom durch definierte Randbedingungen. Clerk Maxwell beschrieb wie groß die elektrischen und magnetischen Felder sind und damit die wirkenden Kräfte.


\textbf{Maxwell-Gleichungen:}

Gaußscher Satz für Elektrizität		\[div \vec{E} = \frac{\rho}{\epsilon}\]		
Gaußscher Satz für Magnetismus 		\[div \vec{B} = 0\]
Faradaysches Gesetz					\[rot \vec{E} = -\frac{\partial\vec{B}}{\partial t}\]
Ampere-Maxwellsches Gesetz			\[rot \vec{B} = \mu\vec{j}+\epsilon\mu\frac{\partial\vec{E}}{\partial t}\]
\cite{halliday2017halliday}

Die dabei auftretenden Konstanten $\epsilon$ (elektrische Feldkonstante) und $\mu$ (magnetische Feldkonstante) sind im SI-Einheitensystem definiert. Die Abkürzungen div und rot stehen für Divergenz und Rotation. 

\subsection{Eigenschaften einer elektromagnetischen Welle}
%TODO
\subsection{Elektronisches Rauschen}
Rauschen im Allgemeinen ist der Oberbegriff für Störspannungen, die ein Nutzsignal überlagern. Es wird als unerwünschtes Signal bezeichnet, das in Kommunikationssystem leider allgegenwärtig ist. Das Rauschen behindert den Empfang von Nachrichten bei der Übertragung, deshalb ist es erforderlich diesen Nebeneffekt zu begrenzen. Rauschen kann mehrere Ursachen haben, die alle durch physikalische Gesetzmäßigkeiten begründet werden können. Üblicherweise sind  Rauschen zufällig auftretende Störspannungen, die keine  Phasen oder Frequenzbeziehung zueinander haben.
%TODO






\newpage
\section{Nachrichten- und Übertragungstechnik}
Die Nachrichten- und Übertragungstechnik beschäftigt sich mit dem Übertragen elektronischer Nachrichten. Als Nachricht wird in diesem Kontext ein vom Sender gezielt erzeugtes Signal verstanden, welches mit Informationen behaftet und für den Empfänger der Nachricht bestimmt ist \cite[vgl. Werner, S. 3]{Werner:2006}.

Das in Abb. \ref{nachrichtenuebertragung} dargestellte Kommunikationsmodell nach Shannon beschreibt den Grundlegenden Aufbau eines Nachrichtenaustausches zweier Systeme. 
Im Folgenden werden die zum weiteren Verständnis notwendigen Begriffe eingeführt:\newline
Die Informationsquelle (\enquote{Information Source}) übergibt die Nachricht dem Sender (\enquote{Transmitter}), der diese mit einem Signal als physikalischem Träger der Nachricht über einen Kanal (\enquote{Channel}) sendet. \newline
Als Kanal bezeichnet man in der Nachrichtentechnik sämtliche technische Komponenten, welche eine Information vom Sender zum Empfänger transportiert. \cite[vgl. Dankmeier, S. 13]{Dankmeier:2017}.\newline
Die im Kanal auftretenden Störsignale, hier durch die Störsignalquelle (\enquote{Noise Source}) dargestellt, überlagern sich mit dem ursprünglichen Signal.\newline
Aus dem für den Empfänger (\enquote{Receiver}) bestimmten Empfangssignal (\enquote{Received Signal}) wird anschließend wieder eine Nachricht generiert, die im letzten Schritt der Informationssenke (\enquote{Destination}) übergeben wird.

\begin{figure}[ht]
	\centering
	\includegraphics[width=\textwidth]{nachrichtenuebertragung-shannon.png}
	\caption[Kommunikationsmodell nach Shannon]{Kommunikationsmodell nach Shannon. Quelle: \cite[Werner, S. 11f]{Werner:2017}} 
	\label{nachrichtenuebertragung}
\end{figure}

Die von der Informationsquelle zu übertragenden Daten werden, je nach Datenquelle, durch eine Quellcodierung komprimiert und beim Empfänger dann durch die entsprechende Umkehroperation die ursprünglichen Daten wiederhergestellt.









\newpage
\section{Signale und Spektren}
Signale werden üblicherweise auf 2 Arten dargestellt:
\begin{enumerate}
	\item Als \textit{Signal} im Zeitbereich
	\item Als \textit{Spektrum} im Frequenzbereich
\end{enumerate}




\subsection{Kontinuierliche und diskrete Signale}
Als Signal gilt eine Funktion mit mindestens einer unabhängigen Variablen, beispielsweise der Zeit \(t\). Ist die Zeitvariable nur für diskrete Werte definiert, so spricht man von einem zeitdiskreten Signal. Man schreibt auch \(x[n]\), wobei \(n\) die normierte Laufvariable genannt wird \cite[vgl. Werner, S. 24]{Werner:2017}.

In Abb. \ref{kontinuierlich_diskret} wird das kontinuierliche Signal
\[x(t) = \sin \omega t = \sin 2\pi f t\]
mit \(f = 50 \text{ Hz} \) dargestellt und mit dem diskreten Signal
\[x[n] = x(nT_a) = \sin 2\pi f T_a n\]
überlagert, welches mit einer Abtastrate von \(f_a = 1 \text{ kHz}\) oder anders ausgedrückt: einem Abtastintervall von \(T_a = 1 / f_a = 1 \text{ ms}\) abgetastet wird \cite[vgl. Heuberger, e. a., S. 11f]{Heuberger:2017}.

\begin{figure}[ht]
	\centering
	\includegraphics[width=\textwidth]{kontinuierlich-diskret.png}
	\caption[Sinus-Signal als kontinuierliches und als diskretes Signal]{Sinus-Signal als kontinuierliches und als diskretes Signal. \newline Quelle: \cite[Heuberger, e. a., S. 12]{Heuberger:2017}} 
	\label{kontinuierlich_diskret}
\end{figure}
%\subsection{Analoge und digitale Signale}
%Digitale Signale sind sowohl werte- als auch zeitdiskret.




\subsection{Nyquist-Shannon-Abtasttheorem}
Das Abtasttheorem nach Nyquist und Shannon besagt, dass ein Signal der Funktion \(x(t)\) welches die Frequenz \(f\) besitzt, mindestens mit der Abtastrate \(f_a\) abgetastet werden muss, wobei gilt: \(f_a \ge 2f\), damit aus dem abgetasteten Signal das Originale durch eine Interpolation ausreichend genau beschrieben werden kann \cite[vgl. Werner, S. 30]{Werner:2006}. 
Ist dies nicht der Fall, kann beim Abtasten zweier Signale unterschiedlicher Frequenzen dazu kommen, dass die resultierenden diskreten Signale identisch sind. Dieser Effekt wird als \enquote{Aliasing} bezeichnet:

\begin{figure}[ht]
	\centering
	\includegraphics[width=0.9\textwidth]{aliasingsines.png}
	\caption[Aliasing Effekt. Abtastung zweier Sinussignale verschiedener Frequenzen]{Aliasing Effekt. Abtastung zweier Sinussignale verschiedener Frequenzen. \newline Quelle: \cite[Moxfyre]{aliasingsampling:2009}} 
	\label{abtasttheorem}
\end{figure}

In der Praxis wird daher meist eine Abtastrate verwendet, die ausreichend größer ist, als der doppelte Betrag der höchsten Frequenz im gegebenen Signal.





\subsection{Spektrum eines Signals}
Eine alternative Darstellung von Signalen kann im Frequenzbereich erfolgen. Dort wir ein Signal mit einzelnen Sinus-Schwingungen beschrieben, aus denen es sich zusammensetzen lässt \cite[vgl. Karrenberg, S. 42]{Karrenberg:2017}.





\subsection{Fourier-Transformation}
Um ein Signal aus dem Zeitbereich in den Frequenzbereich zu überführen wird die sogenannte Fourier-Transformation verwendet.\newline
Im wesentlichen wird ein Signal bzw. eine Funktion mittels Fourier-Transformation als Summe mehrerer Sinus-Schwingungen unterschiedlicher Frequenzen und Amplituden dargestellt.

Signale können also entweder aus Sinusschwingungen konstruiert, oder in solche zerlegt werden.
Nach der Zerlegung ergeben sich einige Möglichkeiten mit den einzelnen Frequenzen zu arbeiten:
\begin{itemize}
	\item Aus einem Signal können einzelne Frequenzen hervorgehoben werden
	\item Bei Audiosignalen beispielsweise können ungewollte Frequenzen ausgeblendet werden (etwa Hintergrundrauschen)
\end{itemize}

Um von dem resultierenden Spektrum wieder zu einer Darstellung im Zeitbereich zu gelangen, kann die Umkehroperation, die inverse Fourier-Transformation angewandt werden.

In der Praxis wird für die Überführung in den Frequenzbereich meist die \ac{FFT} genutzt \cite[vgl. Heuberger, e. a., S. 14]{Heuberger:2017}. Die \ac{FFT} ist allerdings nur eine effizientere Berechnungsart der \ac{DFT}, welche die Fouriertransformation auch mit diskreten Werten ermöglicht \cite[vgl. Meyer, S. 175]{Meyer:2017}.

Um ein Signal mittels \ac{FFT} in den Frequenzraum überführen zu können, muss es die Länge der Form \(N = 2^L\) aufweisen. \(N\) muss also eine 2er-Potenz sein \cite[vgl. Heuberger, e. a., S. 15]{Heuberger:2017}.
Für das zu überführende Signal gilt also:
\[\underline{x} = \Big[ \:  \underline{x}[0] \; \underline{x}[1] \; \underline{x}[2] \; ... \; \underline{x}[N - 2] \; \underline{x}[N - 1] \; \Big] \]







\subsubsection{Fensterfunktionen}
In der Praxis, so auch bei \ac{SDR}-Systemen, handelt es sich in der Regel um Ausschnitte eines Signales. Bei der \ac{FFT} wird mit periodischen Signalen gearbeitet, ein Signalausschnitt ist aber nur quasiperiodisch. Denn an den Rändern gibt es bei der periodischen Fortsetzung des Signals Sprungstellen \cite[vgl. Meyer, S. 187]{Meyer:2017}.\newline
Wird also ein Signal, dessen Länge kein vielfaches einer ganzen Periode ist, aufgezeichnet, entsteht durch die Unstetigkeiten am Rand eine spektrale Streuung durch die Umwandlung mit der \ac{FFT}.

Der Signalausschnitt wird deshalb mit einer sogenannten Fensterfunktion gewichtet: 
\[w = \Big[ \:  w[0] \; w[1] \; w[2] \; ... \; w[N - 2] \; w[N - 1] \; \Big] \]


Die gefensterte FFT Funktion:
\[\text{FFT} _w \: {\underline{x}[n]} = \sum_{n=0}^{N-1} w[n] \: \underline{x}[n] \: e^{-j2\pi nm / N} \text{ mit } m = 0, ..., N-1\]


Das Spektrum dieser Funktion lässt sich als den gewichteten Betrag des resultierenden Ergebnisses ausdrücken \cite[vgl. Heuberger, e. a., S. 14]{Heuberger:2017}:
\[s_x[m] = \frac{1}{c_w^2} \: \Big{|}\underline{X}[m]  \Big{|}^2 \text{ mit } c_w = \sum_{n=0}^{N-1} w[n] \]

In Abbildung \ref{fft} wird das diskrete Zeitsignal \(x[n]\) dem gefensterten Zeitsignal \(w[n]x[n]\) und dem aus der FFT resultierenden Spektrum \(S_x[m]\) gegenübergestellt:
\begin{figure}[ht]
	\centering
	\includegraphics[width=\textwidth]{FFT.png}
	\caption[Zeitsignal, gefenstertes Zeitsignal und Spektrum eines diskreten Sinus-Signals]{Zeitsignal, gefenstertes Zeitsignal und Spektrum eines diskreten Sinus-Signals. Quelle: \cite[Heuberger, e. a., S. 16]{Heuberger:2017}} 
	\label{fft}
\end{figure}




\section{Basisbandübertragung und Trägersignale}
Die Übertragung von Signalen kann über verschiedene Medien erfolgen, etwa metallische- und Lichtwellenleiter oder über Funk. Eine Übertragung im sogenannten Basisband ist jedoch nur mit metallischen Leitern möglich \cite[vgl. Read, S. 149]{Read:2004}. Diese haben den Vorteil, dass ein Signal zur Übertragung den gesamten Frequenzbereich uneingeschränkt nutzen kann. Dies bedeutet, dass das volle Spektrum des Signals in den Übertragungskanal (z. B. Telefonleitung, LAN Kabel) eingespeist werden kann. Eine Verschiebung von den Frequenzen informationstragender Signale ist also nicht notwendig \cite[vgl. Werner, S. 158]{Werner:2006}. Ist das der Fall, so spricht man von einer Basisbandübertragung.\newline


\subsection{Modulationsverfahren}
Signale, die mit einer Funkübertragungstechnik gesendet werden, müssen vor der Übertragung jedoch an Frequenz und Bandbreite des entsprechenden Kanals angepasst werden \cite[vgl. Read, S. 24]{Read:2004}. 
Das hat unter anderem damit zu tun, dass für verschiedene Anwendungen spezielle Bereiche im Frequenzspektrum reserviert sind, siehe Abschnitt \ref{section-frequenzbereiche}.
Des weiteren ist es zum Senden und Empfangen von Signalen vorteilhaft, wenn die Größe der Antenne in etwa der halben Wellenlänge des Signals entspricht \cite[vgl. Kark, S. 4]{Kark:2017}.
Bei der Übertragung von Sprachsignalen, welche (bei Männern) durchschnittlich bei etwa 120 Hz liegen, wäre zum Senden eine Antenne der Größe 1.250 km ideal:
\[ \frac{\lambda}{2} = \frac{c}{2f} = \frac{3*10^8 \text{ m/s}}{2*120 \text{ Hz}} = 1.250.000 \text{ m}\]

Da eine Antenne dieser Größe utopisch ist, können die zu übertragenden Informationen einfach auf höhere Frequenzen (mit entsprechend kleineren Wellenlängen) verschoben werden. Hierbei spricht man von \textit{Modulation}.
Die informationstragenden Signale werden dann bevorzugt einer elektromagnetischen, sinusförmigen Welle über die Amplitude, der Phase und/oder der Frequenz aufgeprägt \cite[vgl. Werner, S. 242f]{Werner:2017}. %TODO: "... der Phase und/oder der Frequenz" 

Bei der Modulation wird der Frequenzbereich des Basisbandes, also des zu modulierenden Signals, in einen anderen Frequenzbereich transformiert \cite[vgl. Plaßmann, S. 1204]{Plassmann:2016}.


\subsubsection{I/Q-Modulation}




\newpage
\section{Frequenzbereiche}
\label{section-frequenzbereiche}
Zur Orientierung im Spektrum elektromagnetischer Wellen haben sich international verschiedene Systeme zur Klassifikation sogenannter Frequenzbänder gebildet. Die \ac{ITU} empfiehlt eine Einteilung des Spektrums von 3 kHz bis 300 GHz in acht Frequenzbereiche, auch Frequenzdekaden genannt. \cite[vgl. ITU-R v.431-8]{itu-431:2015}

\begin{figure}[ht]
	\centering
	\includegraphics[width=0.75\textwidth]{frequenzbereich.png}
	\caption[Spektrum elektromagnetischer Wellen und gebräuchliche Bandbezeichnungen]{Spektrum elektromagnetischer Wellen und gebräuchliche Bandbezeichnungen. Quelle: \cite[Kark, S. 1]{Kark:2017}} 
	\label{frequenzbereiche}
\end{figure}






\subsection{Rechtliche Grundlagen} %TODO
In Deutschland gilt rechtlich zudem die Aufteilung des Frequenzbereiches von 9 kHz bis 3000 GHz, welche von der Bundesnetzagentur im sogenannten Frequenzplan \cite[Bundesnetzagentur, 2016]{bundesnetzagentur-frequenzplan:2016} gemäß § 54 TKG festgehalten wird.
Dort werden die Frequenzbereiche nach Frequenznutzung (Amateurfunk, Seefunk, WLAN, etc.) eingeteilt und entsprechende Nutzungsbestimmungen spezifiziert:

\begin{figure}[ht]
	\centering
	\includegraphics[width=\textwidth]{freqplan-wlan.png}
	\caption[Eintrag: 2,4 GHz WLAN im Frequenzplan]{Eintrag: 2,4 GHz WLAN im Frequenzplan. Quelle: \cite[Bundesnetzagentur, 2016]{bundesnetzagentur-frequenzplan:2016}}
	\label{frequenzplan-wlan}
\end{figure}







\section{Anwendungen im Bereich der Dezimeterwelle}
Das Frequenzband von 300 MHz bis 3 GHz, auch \ac{UHF}-Band genannt, ist ein Frequenzbereich in dem die Wellen eine Länge von zehn Dezimeter bis einem Dezimeter besitzen:
\( \lambda = \frac{c}{f} = \frac{3*10^8 \text{ m/s}}{300 \text{ MHz}} = 1\text{ m}\)
und
\( \frac{3*10^8 \text{ m/s}}{3000 \text{ MHz}} = 0.1\text{ m}\)






\subsection{Bluetooth}
Bluetooth ist eine Übertragungstechnik für kabellose Kommunikation über kurze Distanzen. Es wird im Frequenzbereich von 2,4 bis 2,4835 GHz betrieben \cite[Bundesamt für Strahlenschutz, S. 1]{bundesamt-strahlungsschutz:2012}. Insgesamt gibt es unter Bluetoothgeräten drei verschiedene Sendeleistungsklassen:
\begin{description}
	\item[Klasse 1: bis 1,0 mW] Reichweite: bis 10m 
	\item [Klasse 2: bis 2,5 mW] Reichweite: 10m und mehr
	\item [Klasse 3: bis 100 mW] Reichweite: 100m und mehr
\end{description}
Die Aufteilung des Frequenzbereiches von 0 kHz bis 3000 GHz wird von der Bundesnetzagentur im sogenannten Frequenzplan \cite[Bundesnetzagentur]{bundesnetzagentur-frequenzplan:2016} gemäß § 54 TKG festgehalten.





\subsection{Wireless Local Area Network}
Unter dem Begriff Wireless Local Network (WLAN) versteht man ein kabelloses lokales Netzwerk, welches meist an Orten eingesetzt wird, bei der kabelgebundene Datenübertraung zu teuer, umständlich oder umkomfortabel wäre.






\section{Software Defined Radio Systeme} 
\ac{SDR} Systeme sind digitale Datenübertragungssysteme, bei denen der Großteil der Signal- und Datenverarbeitung mittels Softwarekomponenten erfolgt \cite[Heuberger, e. a., S. 1]{Heuberger:2017}.
Besonders hervorzuheben ist bei \ac{SDR} Systemen, dass ihre Hardware größtenteils unabhängig nachrichtentechnischer Eigenschaften wie der Symbolrate und Modulationsart sind. \ac{SDR} Systeme können verschiedene Standards verwenden um entsprechende Übertragungsarten zu implementieren, da die Funktionalität durch Software, also Hardware-unabhängig, realisiert wird \cite[Heuberger, e. a., S. 36]{Heuberger:2017}. \newline
Abbildung \ref{sdr-blockschaltbild} zeigt den Aufbau eines \ac{SDR} Systems als Blockschaltbild, in dem sowohl Sender und Empfänger als SDR realisiert sind.
Für jedes Element im Sender gibt es ein entsprechendes Element im Empfänger, welches die Senderoperationen rückgängig macht, um die ursprünglichen Daten rekonstruieren zu können.

%TODO: Hab ich mal so formuliert, aber keine Ahnung ob das auch so bleibt oder genau zutrifft. -> Evtl. umformulieren!
Da sich diese Studienarbeit mit dem Empfangen und Auswerten von Signalen, aber nicht dem Senden befasst, wird im weiteren Verlauf des Dokumentes hauptsächlich die Seite des Empfängersystems und der Übertragungsstrecke betrachtet.

\begin{figure}[ht]
	\centering
	\includegraphics[width=0.9\textwidth]{sdr.png}
	\caption[Blockschaltbild eines SDR Systems]{Blockschaltbild eines SDR Systems. Quelle: \cite[Heuberger, e. a., S. 37]{Heuberger:2017}}
	\label{sdr-blockschaltbild}
\end{figure}

Abbildung \ref{nachrichtentechn-blockschaltbild} zeigt den Aufbau eines klassischen digitalen Übertragungssystems. Die Funktionsweise lässt sich wie folgt zusammenfassen:
\begin{enumerate} %TODO: Vervollständigen!
	\item Um später die Integrität der digitalen Daten verifizieren zu können, werden sie mit einer \ac{CRC}-Kodierung versehen.
	\item Kanalcodierung %TODO
	\item Scrambler %TODO
	\item Symbol-Mapper %TODO
	\item Das Signal wird einem Trägersignal aufgeprägt, damit es übertragen werden kann.
\end{enumerate}

Anschließend erfolgen auf der Empfängerseite die entsprechenden inversen Operationen zu den aufgeführten Schritten. In einem Software Defined Radio System werden \textit{alle} diese Vorgänge durch Software in der Basisbandverarbeitung abgebildet \cite[vgl. Heuberger, e. a., S. 38]{Heuberger:2017}

\begin{figure}[ht]
	\centering
	\includegraphics[width=\textwidth]{sdr-nachrichtentechn-blockschaltbild.png}
	\caption[Nachrichtentechnisches Blockschaltbild eines digitalen Übertragungssystems]{Nachrichtentechnisches Blockschaltbild eines digitalen Übertragungssystems. Quelle: \cite[Heuberger, e. a., S. 38]{Heuberger:2017}}
	\label{nachrichtentechn-blockschaltbild}
\end{figure}






