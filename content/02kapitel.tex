%!TEX root = ../dokumentation.tex

\chapter{Theoretische Grundlagen}
\section{Elektromagnetische Wellen}

Als elektromagnetische Welle werden räumlich ausbreitende Wellen bezeichnet, die aus elektrischen und magnetischen Feldern besteht. Die laufende Welle breitet sich entlang der Ausbreitungsgeschwindigkeit mit einer Geschwindigkeit von \[c= 3*10^8 m/s\] aus. Die elektrischen und magnetischen Feldvektoren der Welle stehen orthogonal aufeinander, wie in der Abbildung \ref{elektromagnetische Welle} gut zu erkennen.

\begin{figure}[ht]
	\centering
	\includegraphics[width=0.75\textwidth]{em-welle.png}
	\caption[Feldvektoren der elektromagnetischen Welle]{Feldvektoren der elektromagnetischen Welle} 
	\label{elektromagnetische Welle}
\end{figure}

Das Spektrum einer solchen elektromagnetischen Welle reicht von langsamen Radiowellen, Infrarotwellen über den Bereich des sichtbaren Lichts bis hin zur Röntgenstrahlung und der extrem kurzwelligen Gammastrahlung.
Die elektromagnetischen Wellen, die auf unsere Erde einwirken werden auch natürliche Strahlung genannt. Sie ermöglicht indirekt das Leben auf der Erde, durch die Energie Zufuhr der Sonne über die infraroten Wellen.
Elektromagnetische Wellen benötigen kein Medium, um sich auszubreiten. Sie können sich daher auch über weiteste Entfernungen ausbreiten. Sie bewegen sich im Vakuum unabhängig von ihrer Frequenz mit Lichtgeschwindigkeit fort. Elektromagnetische Wellen können sich aber auch in Materie ausbreiten wie etwa Gas oder Flüssigkeit, jedoch verringert sich dabei ihre Geschwindigkeit.

\subsection{Eigenschaften einer elektromagnetischen Welle}

\subsection{Elektronisches Rauschen}
Rauschen im Allgemeinen ist der Oberbegriff für Störspannungen, die ein Nutzsignal überlagern. Rauschen kann mehrere Ursachen haben, die alle durch physikalische Gesetzmäßigkeiten begründet werden können. 


\newpage
\section{Nachrichten- und Übertragungstechnik}
Die Nachrichten- und Übertragungstechnik beschäftigt sich mit dem Übertragen elektronischer Nachrichten. Als Nachricht wird in diesem Kontext ein vom Sender gezielt erzeugtes Signal verstanden, welches mit Informationen behaftet und für den Empfänger der Nachricht bestimmt ist \cite[vgl. Werner, S. 3]{Werner:2006}.

Das in Abb. \ref{nachrichtenuebertragung} dargestellte Kommunikationsmodell nach Shannon beschreibt den Grundlegenden Aufbau eines Nachrichtenaustausches zweier Systeme. 
Im Folgenden werden die zum weiteren Verständnis notwendigen Begriffe eingeführt:\newline
Die Informationsquelle (\enquote{Information Source}) übergibt die Nachricht dem Sender (\enquote{Transmitter}), der diese mit einem Signal als physikalischem Träger der Nachricht über einen Kanal (\enquote{Channel}) sendet. \newline
Als Kanal bezeichnet man in der Nachrichtentechnik sämtliche technische Komponenten, welche eine Information vom Sender zum Empfänger transportiert. \cite[vgl. Dankmeier, S. 13]{Dankmeier:2017}.\newline
Die im Kanal auftretenden Störsignale, hier durch die Störsignalquelle (\enquote{Noise Source}) dargestellt, überlagern sich mit dem ursprünglichen Signal. Aus dem für den Empfänger (\enquote{Receiver}) bestimmten Empfangssignal (\enquote{Received Signal}) wird anschließend wieder eine Nachricht generiert, die im letzten Schritt der Informationssenke (\enquote{Destination}) übergeben wird.

\begin{figure}[ht]
	\centering
	\includegraphics[width=\textwidth]{nachrichtenuebertragung-shannon.png}
	\caption[Kommunikationsmodell nach Shannon]{Kommunikationsmodell nach Shannon. Quelle: \cite[Werner, S. 11f]{Werner:2017}} 
	\label{nachrichtenuebertragung}
\end{figure}

\section{Signale und Spektren}
Signale werden üblicherweise auf 2 Arten dargestellt:
\begin{enumerate}
	\item Als \textit{Signal} im Zeitbereich
	\item Als \textit{Spektrum} im Frequenzbereich
\end{enumerate}

\subsection{Kontinuierliche und diskrete Signale}
Als Signal gilt eine Funktion mit mindestens einer unabhängigen Variablen, beispielsweise der Zeit \(t\). Ist die Zeitvariable nur für diskrete Werte definiert, so spricht man von einem zeitdiskreten Signal. Man schreibt auch \(x[n]\), wobei \(n\) die normierte Laufvariable genannt wird \cite[vgl. Werner, S. 24]{Werner:2017}.

In Abb. \ref{kontinuierlich_diskret} wird das kontinuierliche Signal
\[x(t) = \sin \omega t = \sin 2\pi f t\]
mit \(f = 50 \text{ Hz} \) dargestellt und mit dem diskreten Signal
\[x[n] = x(nT_a) = \sin 2\pi f T_a n\]
überlagert, welches mit einer Abtastrate von \(f_a = 1 \text{ kHz}\) oder anders ausgedrückt: einem Abtastintervall von \(T_a = 1 / f_a = 1 \text{ ms}\) abgetastet wird \cite[vgl. Heuberger, e. a., S. 11f]{Heuberger:2017}.

\begin{figure}[ht]
	\centering
	\includegraphics[width=\textwidth]{kontinuierlich-diskret.png}
	\caption[Sinus-Signal als kontinuierliches und als diskretes Signal]{Sinus-Signal als kontinuierliches und als diskretes Signal. \newline Quelle: \cite[Heuberger, e. a., S. 12]{Heuberger:2017}} 
	\label{kontinuierlich_diskret}
\end{figure}
%\subsection{Analoge und digitale Signale}
%Digitale Signale sind sowohl werte- als auch zeitdiskret.

\subsection{Spektrum eines Signals}


\section{Nyquist-Shannon-Abtasttheorem}

\newpage
\section{Frequenzbereiche}
Zur Orientierung im Spektrum elektromagnetischer Wellen haben sich international verschiedene Systeme zur Klassifikation sogenannter Frequenzbänder gebildet. Die \ac{ITU} empfiehlt eine Einteilung des Spektrums von 3 kHz bis 300 GHz in acht Frequenzbereiche, auch Frequenzdekaden genannt. \cite[vgl. ITU-R v.431-8]{itu-431:2015}

\begin{figure}[ht]
	\centering
	\includegraphics[width=0.75\textwidth]{frequenzbereich.png}
	\caption[Spektrum elektromagnetischer Wellen und gebräuchliche Bandbezeichnungen]{Spektrum elektromagnetischer Wellen und gebräuchliche Bandbezeichnungen. Quelle: \cite[Kark, S. 1]{Kark:2017}} 
	\label{frequenzbereiche}
\end{figure}


\subsection{Rechtliche Grundlagen} %TODO
In Deutschland gilt rechtlich zudem die Aufteilung des Frequenzbereiches von 9 kHz bis 3000 GHz, welche von der Bundesnetzagentur im sogenannten Frequenzplan \cite[Bundesnetzagentur, 2016]{bundesnetzagentur-frequenzplan:2016} gemäß § 54 TKG festgehalten wird.
Dort werden die Frequenzbereiche nach Frequenznutzung (Amateurfunk, Seefunk, WLAN, etc.) eingeteilt und entsprechende Nutzungsbestimmungen spezifiziert:

\begin{figure}[ht]
	\centering
	\includegraphics[width=\textwidth]{freqplan-wlan.png}
	\caption[Eintrag: 2,4 GHz WLAN im Frequenzplan]{Eintrag: 2,4 GHz WLAN im Frequenzplan. Quelle: \cite[Bundesnetzagentur, 2016]{bundesnetzagentur-frequenzplan:2016}}
	\label{frequenzplan-wlan}
\end{figure}

\subsection{Dezimeterwelle}
Das Frequenzband von 300 MHz bis 3 GHz, auch \ac{UHF}-Band genannt, ist ein Frequenzbereich in dem die Wellen eine Länge von zehn Dezimeter bis einem Dezimeter besitzen.

\section{Bluetooth}
Bluetooth ist eine Übertragungstechnik für kabellose Kommunikation über kurze Distanzen. Es wird im Frequenzbereich von 2,4 bis 2,4835 GHz betrieben \cite[Bundesamt für Strahlenschutz, S. 1]{bundesamt-strahlungsschutz:2012}. Insgesamt gibt es unter Bluetoothgeräten drei verschiedene Sendeleistungsklassen:
\begin{description}
	\item[Klasse 1: bis 1,0 mW] Reichweite: bis 10m 
	\item [Klasse 2: bis 2,5 mW] Reichweite: 10m und mehr
	\item [Klasse 3: bis 100 mW] Reichweite: 100m und mehr
\end{description}
Die Aufteilung des Frequenzbereiches von 0 kHz bis 3000 GHz wird von der Bundesnetzagentur im sogenannten Frequenzplan \cite[Bundesnetzagentur]{bundesnetzagentur-frequenzplan:2016} gemäß § 54 TKG festgehalten.

\section{Wireless Local Area Network}
Unter dem Begriff Wireless Local Network (WLAN) versteht man ein kabelloses lokales Netzwerk, welches meist an Orten eingesetzt wird, bei der kabelgebundene Datenübertraung zu teuer, umständlich oder umkomfortabel wäre.

\section{Software Defined Radio} 
\enquote{Funkübertragungssysteme, bei denen wesentliche Teile der Verarbeitung mittels Software erfolgen, werden als Software Defined Radio \ac{SDR}-Systeme bezeichnet.} \cite[Heuberger, e. a., S. 1]{Heuberger:2017}


