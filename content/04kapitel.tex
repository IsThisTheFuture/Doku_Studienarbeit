%!TEX root = ../dokumentation.tex

\chapter{Praxis Kapitel}
In diesem Kapitel wird die softwareseitige Umsetzung der Studienarbeit beschrieben. Zuerst wird auf die Einrichtung der Entwicklungsumgebung eingegangen und sichergestellt, dass die installierte Hardware korrekt funktioniert. Anschließend wird ein beispielhafter GNU Radio Companion Flowgraph erstellt um sich mit dem Programm vertraut zu machen und die Grundlage für das zu entwickelnde Analyseprogramm zu schaffen.\\
Aus den gewonnen Erkenntnissen wird eine QT-GUI Bedienmaske erstellt, welche es ermöglicht die eingangs beschriebenen Metainformationen zu visualisieren. Die Oberfläche bietet ebenfalls entsprechende \enquote{Stellschrauben} um das zu betrachtende Signal im jeweiligen Anwendungskontext zu betrachten. 


\section{Einrichten der Entwicklungsumgebung}
Für den täglichen Gebrauch ist es nicht empfehlenswert Binaries als Root-User auszuführen. Um das HackRF One mit einem regulären Nutzer auf einem Linux System nutzen zu können, ist es notwendig eine entsprechende udev-Regel für das Gerät zu schreiben:

\begin{lstlisting}[caption=Erstellen einer udev-Regel, label=udev]
ATTR{idVendor}=="1d50", ATTR{idProduct}=="6089", SYMLINK+="hackrf-one-%k", MODE="660", GROUP="plugdev"
\end{lstlisting}

Die Datei mit der Regel wird unter /etc/udev/rules.d/ abgelegt. Verbindet man das Gerät nun mit dem Computer, wird ein Symlink unter /dev/ angelegt und Mitglieder der Gruppe \textit{plugdev} haben Zugriff darauf. 
Die erfolgreiche Installation kann durch das Ausführen des Kommandos \textit{hackrf\_info} als User ohne root-Berechtigung überprüft werden:
\begin{lstlisting}[caption=Das Kommando "hackrf\_info" wird zum Test auf dem System ausgeführt, label=hackrfinfo]
hackrf_info version: git-a4c57ef
libhackrf version: git-a4c57ef (0.5)
Found HackRF
Index: 0
Serial number: 0000000000000000a06063c824237d5f
Board ID Number: 2 (HackRF One)
Firmware Version: 2017.02.1 (API:1.02)
Part ID Number: 0xa000cb3c 0x00544765
\end{lstlisting}

\newpage
Um sicherzustellen, dass das Hostsystem mit der vollen Samplerate des HackRF One umgehen kann, wird das Binary \textit{hackrf\_transfer} ausgeführt und entsprechend parametrisiert:

\begin{lstlisting}[caption=Empfangen von Daten mit hackrf\_transfer, label=hackrf-receive]
$ hackrf_transfer -r /tmp/hackRF-20M.received -f 2400000 -l 16 -g 16 -s 20000000
# -r := Recieve    -f := Zielfrequenz    -l := RX LNA (IF) gain    
# -g := RX VGA (baseband) gain   -s := Sample Rate (Hz)
call hackrf_set_sample_rate(20000000 Hz/20.000 MHz)
call hackrf_set_hw_sync_mode(0)
call hackrf_set_freq(2400000 Hz/2.400 MHz)
Stop with Ctrl-C
39.8 MiB / 1.000 sec = 39.8 MiB/second
40.1 MiB / 1.001 sec = 40.1 MiB/second
40.1 MiB / 1.000 sec = 40.1 MiB/second
39.8 MiB / 1.000 sec = 39.8 MiB/second
40.1 MiB / 1.000 sec = 40.1 MiB/second
39.8 MiB / 1.000 sec = 39.8 MiB/second
40.1 MiB / 1.000 sec = 40.1 MiB/second
^CCaught signal 2
17.3 MiB / 0.435 sec = 39.7 MiB/second

Exiting...
Total time: 7.43705 s
hackrf_stop_rx() done
hackrf_close() done
hackrf_exit() done
fclose(fd) done
exit
\end{lstlisting}

Das ausgeführte Kommando nutzt die HackRF-Bibliothek zum Ansprechen des Gerätes. 
Mit \textit{hackrf\_transfer} kann sowohl empfangen, als auch gesendet werden. 
Hier wird die Frequenz \(f = 2,4\) GHz mit einer Abtastrate  \( f_a = 20 \) MHz gesampled.
Das Ergebnis wird in einer Datei abgespeichert. \\
Es ist zu erkennen, dass das Gerät die maximale Abtastrate

\newpage
\section{Grundbausteine in GNU Radio Companion}
GNU Radio Companion wird am besten durch das klassische FM-Radio Beispiel erklärt:

\begin{figure}[ht]
	\centering
	\includegraphics[width=\textwidth]{fmradio.png}
	\caption[Implementation eines FM-Radioempfängers in GNU Radio Companion]{Implementation eines FM-Radioempfängers in GNU Radio Companion. Quelle: Eigene Darstellung} 
	\label{fmradio}
\end{figure}

Das dargestellte Blockschaltbild realisiert das Empfangen, Demodulieren und Wiedergeben eines FM-Radio Signals. Das Programm bietet zudem die Möglichkeit Frequenz und Lautstärke anzupassen:

\begin{figure}[ht]
	\centering
	\includegraphics[width=0.8\textwidth]{fmradio-fft.png}
	\caption[FM Radio FFT-Plot]{FM Radio FFT-Plot. Quelle: Eigene Darstellung} 
	\label{fmradio-fft}
\end{figure}

Im Folgenden werden die einzelnen Bausteine näher erläutert.


\newpage
Als Signalquelle können in GNU Radio TCP/UDP Ports, Soundkarten, Dateien bzw. Input-Streams oder \ac{SDR}-Geräte benutzt werden.\newline
Um das HackRF One als Signalquelle zu verwenden, wird der quelloffene, von \ac{osmocom} \cite{osmocom:2018} entwickelte GNU Radio Block \textit{gr-osmosdr} \cite{gr-osmosdr:2018} benutzt.
Dieser untersützt das HackRF durch die zuvor installierte Bibliothek \textit{libhackrf}:

\begin{figure}[ht]
	\centering
	\includegraphics[width=0.5\textwidth]{osmocom-source.png}
	\caption[osmocom Source Block]{osmocom Source Block. Quelle: Eigene Darstellung} 
	\label{osmocom-source}
\end{figure}

\begin{description}
	\item[ID:] Der Name der Python-Variable des GRC Blocks
	\item[Output Type:] Der vom Block produzierte Datentyp, in diesem Fall \textit{Complex float32}, da das HackRF One Samples als IQ-Paare darstellt, wobei die I- und Q-Komponente jeweils als Float-Gleitkommazahl repräsentiert werden
	\item [Sample Rate (sps):] Die Abtastrate $f_a$ in Samples pro Sekunde
	\item[Ch0 Frequency (Hz):] Die Frequenz des Kanals. Da nur eine Antenne vorhanden ist, wird automatisch Channel 0 benutzt
	\item[Ch0 Freq. Corr. (ppm):] description %TODO
	\item[Ch0 RF Gain (dB):] description
	\item[Ch0 IF Gain (dB):] Intermediate Frequency
	\item[Ch0 BB Gain (dB):] Baseband
	\item[Ch0 Bandwidth (Hz):] Bandbreite
\end{description}






\section{Erarbeiten der relevanten Metainformationen}

\subsubsection{Antennengewinn}
\subsubsection{Signalleistung}
\subsubsection{Quantisierung}
\subsubsection{Diskretisierung}
\subsubsection{Frequenz}
\subsubsection{DC Offset}
Durch die Überführung eines modulierten Signals zurück ins Basisband entsteht im FFT-Plot um die Nullfrequenz eine deutlich wahrnehmbare Spitze:
\begin{figure}[ht]
	\centering
	\includegraphics[width=0.6\textwidth]{dc-offset.png}
	\caption[DC Offset]{DC Offset. Quelle: Eigene Darstellung} 
	\label{dc-offset}
\end{figure}

Es gibt verschiedene Wege damit umzugehen:
\begin{enumerate}
	\item Für einige Anwendungen kann das DC Offset einfach vernachlässigt werden.
	\item Das DC Offset lässt sich für viele Signalarten umgehen indem die hohe Bandbreite des HackRF ausgenutzt wird. 
	Dazu stellt man nicht die Zielfrequenz ein, sondern eine außerhalb liegende Frequenz, so dass das eigentlich zu erfassende Signal von 0 Hz  weggeschoben wird. Weil das Signal natürlich vollständig in der erfassten Bandbreite enthalten sein muss, erhöht sich die Menge der zu analysierenden Daten und damit auch die CPU Auslastung. Die Bandbreite des SDR Gerätes sollte also bestenfalls nur so groß sein wie nötig um unnötige Last zu vermeiden.
	\item Das DC Offset lässt sich über Software korrigieren bzw. abschwächen. Das führt unter Umständen dazu dass auch Frequenzen nahe 0 Hz ungewollterweise modifiziert werden. Das Signal wird hierdurch also leicht verfälscht. Einige Spektralanalyseprogramme (GQRX, SDR\#) bieten dieses Feature an.
\end{enumerate}

\begin{figure}[ht]
	\centering
	\includegraphics[width=0.7\textwidth]{dc-offset-correction.png}
	\caption[DC Offset Korrektur]{DC Offset Korrektur. Quelle: Eigene Darstellung} 
	\label{dc-offset-correction}
\end{figure}


\subsubsection{Bandbreite}



\subsection{Empfängerempfindlichkeit}
\subsubsection{Signal-Rausch-Verhältnis}
Das Signal-Rausch-Verhältnis (engl.: \enquote{Signal-to-Noise Ratio}) gibt das Verhältnis eines Nutzsignals 
zu einem Rauschsignal an, von dem es überlagert ist. Das Rauschen kann dabei durch das Empfängersystem selbst, 
sowie durch Störungen im Übertragungskanal verursacht worden sein. %TODO: Es gibt glaub mehrere Definitionen

\[ S/N = \frac{\text{Nutzsignalleistung}}{\text{Rauschsignalleistung}} = \frac{P_{Signal}}{P_{Rauschen}} \]


Für den Fall \(P_{Signal} \leq P_{Rauschen}\), also wenn gilt: 
Die Signalleistung ist kleiner oder gleich der Rauschsignalleistung, kann ein Signal in der Regel nicht erkannt werden.
Damit dies gelingt, muss die Energie des Signals addiert mit der des Rauschsignals größer sein als ein festgelegter Schwellwert.
Der Schwellwert wird wesentlich höher gelegt als das durchschnittliche Rauschniveau, um die beiden deutlich von einander unterscheiden zu können.
Es bietet sich also an das Verhältnis in Dezibel auszudrücken:
\[ S/N = 10 \lg \Big( \frac{P_{Signal}}{P_{Rauschen}} \Big) \text{dB}\] 



Das minimale Signal-Rausch-Verhältnis \(S/N_{min}\) gibt diesen Schwellwert an, der mindestens erreicht werden muss um das Signal zu erkennen. 
%TODO: Nutzsignalleistung und Rauschsignalleistung definieren



\newpage
\section{Identifizieren und Dekodieren eines Signales}
Signale müssen im Empfänger gemittelt werden, da sie aufgrund der durchgeführten Kanalfilterung ein oszillierendes Verhalten aufweisen, siehe Abb. \ref{mod-send-empf} \cite[vgl. Heuberger, e. a., S. TODO]{heuberger:2017}.

\begin{figure}[ht]
	\centering
	\includegraphics[width=0.8\textwidth]{mod-sender-empf.png}
	\caption[GFSK-Modulationssignale im Sender und im Empfänger]{GFSK-Modulationssignale im Sender und im Empfänger. Quelle: TODO} %TODO Selbst darstellen, am besten anhand eines 433MHz Signals! 
	\label{mod-send-empf}
\end{figure}


\newpage
\section{GRC-Flowgraph zur Analyse von Metadaten}
\textbf{TODO: } GRC Flowgraph mit Optionsfeldern (Frequenz-slider, Parameter für FFT: Bin count/size,Signal-To-Noise Ratio, )\\
\\
\textbf{Mögliche Leitfrage: Was sind Metainformationen überhaupt? Welche Davon sind relevant für Software Defined Radio? (Signal-To-Noise, Antennengewinn/Verstärkung/Gain (dB), Quantisierung (Werte, z.B. die Amplitude bei I/Q Signalen), Diskretisierung (zeitlich), Rauschen, etc.)} $\rightarrow$ GRC-Flowgraph erstellen, der diese Metainformationen anzeigt und Einstellungsmöglichkeiten bietet

\begin{figure}[ht]
	\centering
	\includegraphics[width=0.5\textwidth]{metaanalyse.png}
	\caption[GRC-Flowgraph zur Analyse von Metadaten]{GRC-Flowgraph zur Analyse von Metadaten. Quelle: Eigene Darstellung, GNU Radio 3.7.11} 
	\label{metaanalyse}
\end{figure}




