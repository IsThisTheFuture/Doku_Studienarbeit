%!TEX root = ../dokumentation.tex

\chapter{Einleitung}
Diese Studienarbeit befasst sich mit dem Erfassen und Auswerten von Signalen in einem begrenzten Frequenzbereich.  

\section{Problemstellung}
%\begin{description}
%	\item \textbf{Warum ist das Thema relevant? In welchem Kontext?}

%\end{description}	

\textbf{Was ist die Empfängerempfindlichkeit (Receiver Sensitivity) des HackRF?}\\
\url{http://www.phys.hawaii.edu/~anita/new/papers/militaryHandbook/rcvr_sen.pdf}


\textbf{What is the minimum signal power level that can be detected by HackRF?\\
A:}

This isn't a question that can be answered for a general purpose SDR platform such as HackRF. Any answer would be very specific to a particular application. For example, an answerable question might be: What is the minimum power level in dBm of modulation M at frequency F that can be detected by HackRF One with software S under configuration C at a bit error rate of no more than E%? Changing any of those variables (M, F, S, C, or E) would change the answer to the question. Even a seemingly minor software update might result in a significantly different answer. To learn the exact answer for a specific application, you would have to measure it yourself.

HackRF's concrete specifications include operating frequency range, maximum sample rate, and dynamic range in bits. These specifications can be used to roughly determine the suitability of HackRF for a given application. Testing is required to finely measure performance in an application. Performance can typically be enhanced significantly by selecting an appropriate antenna, external amplifier, and/or external filter for the application.

\section{Zielsetzung}
Ziel der Studienarbeit ist es die Frage zu klären, was Metainformationen im Kontext von \ac{SDR} bedeuten und welche Informationen zur Realisierung eines generischen Gerätes zur Analyse einer möglichst hohen Anzahl von Signalen innerhalb der Dezimeterwelle erforderlich sind.\\

Auf dieser Grundlage wird ein System ausgelegt und implementiert, welches die gestellten Anforderungen erfüllt.\\

Die relevanten Metainformationen sollen in einem Programm mit grafischer Oberfläche visualisiert werden. Zudem soll darüber das Empfängersystem gesteuert und das Signal manipuliert werden können.



%\begin{description}
%	\item \textbf{Welche Fragen soll die Arbeit beantworten?}
%	\item \textbf{Welchem Zweck dient die Arbeit?}
%\end{description}


\section{Abgrenzung}


\section{Vorgehensweise}
Zu Beginn werden in Abschnitt 2 die theoretischen Grundlagen geschaffen, die für das Verständnis dieser Projektarbeit nötig sind. \\
Darauf folgt in Kapitel 3 die Anforderungsanalyse, sowie Planung und Aufbau des Empfängersystems mit Funkantenne.\\
In Kapitel 4 wird mit Hilfe von GNU Radio eine grafische Oberfläche gestaltet, über die sich beliebige Signale im gegebenen Frequenzbereich analysieren lassen.\\
Abschließend folgt eine Zusammenfassung mit Fazit und Ausblick.

%\begin{description}
%	\item \textbf{Welche Vorgehensweise wird hierzu gewählt?}
%\end{description}









