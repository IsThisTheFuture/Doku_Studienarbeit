%!TEX root = ../dokumentation.tex

\chapter{Einleitung}
Im Rahmen dieser Studienarbeit soll ein Gerät zur Erfassung von Metainformationen über Signale innerhalb der Dezimeterwelle entwickelt werden. Dabei stützen wir uns auf ein Software Defined Radio Gerät zur digitalen Auswertung dieser Signale.
Es gilt zu klären, was man unter Metainformationen versteht und welche davon im Kontext von Software Defined Radio relevant sind.

\section{Problemstellung}
Im alltäglichen Leben rauschen unzählige Funksignale an uns vorbei. Im Bereich der Dezimeterwelle sind viele gängige Anwendungen angesiedelt, die von Verbrauchern genutzt werden, etwa GSM, GPS, WLAN und Bluetooth. 
Deshalb ist es schwer, genau sagen zu können, was beachtet werden muss, um ein generisches System zur Analyse möglichst vieler dieser Signale zu entwickeln.


\section{Zielsetzung}
Ziel der Studienarbeit ist es die Frage zu klären, was Metainformationen im Kontext von \ac{SDR} bedeuten und welche Informationen zur Realisierung eines generischen Gerätes zur Analyse einer möglichst hohen Anzahl von Signalen innerhalb der Dezimeterwelle erforderlich sind.

Auf dieser Grundlage wird ein System ausgelegt und implementiert, welches die gestellten Anforderungen erfüllt.

Die relevanten Metainformationen sollen in einem Programm mit grafischer Oberfläche visualisiert werden. Zudem soll sich darüber das Empfängersystem steuern lassen. Das Programm ist so zu realisieren, dass es für konkrete Anwendungen erweitert werden kann.


\section{Abgrenzung}
Da die Aufgabenstellung sehr allgemein gehalten ist, verzichten wir darauf einzelne Signale im Detail zu betrachten.
Stattdessen wird der Schwerpunkt dieser Studienarbeit auf die Entwicklung eines Gerätes im Kontext von Signalen im Bereich der Dezimeterwelle beschränkt.

\section{Vorgehensweise}
Zu Beginn werden in Abschnitt 2 die theoretischen Grundlagen geschaffen, die für das Verständnis dieser Studienarbeit nötig sind. \\
Darauf folgt in Kapitel 3 die Anforderungsanalyse, sowie Evaluierung, Planung und der Aufbau des Empfängersystems.\\
In Kapitel 4 wird mit Hilfe der ausgewählten Software eine grafische Oberfläche gestaltet, über die sich beliebige Signale im gegebenen Frequenzbereich analysieren lassen sollen.\\
Im Zuge dieser Arbeit wird abschließend eine Zusammenfassung mit Fazit und Ausblick gegeben.










