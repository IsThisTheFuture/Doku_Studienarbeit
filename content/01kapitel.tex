%!TEX root = ../dokumentation.tex

\chapter{Einleitung}
Diese Studienarbeit befasst sich mit dem Erfassen und Auswerten von Signalen in einem begrenzten Frequenzbereich.  

\section{Problemstellung}
%\begin{description}
%	\item \textbf{Warum ist das Thema relevant? In welchem Kontext?}

%\end{description}	

\section{Zielsetzung}
Ziel der Studienarbeit ist es die Frage zu klären, was Metainformationen im Kontext von \ac{SDR} bedeuten und welche Informationen zur Realisierung eines generischen Gerätes zur Analyse einer möglichst hohen Anzahl von Signalen innerhalb der Dezimeterwelle erforderlich sind.\\

Auf dieser Grundlage wird ein System ausgelegt und implementiert, welches die gestellten Anforderungen erfüllt.\\

Die relevanten Metainformationen sollen in einem Programm mit grafischer Oberfläche visualisiert werden. Zudem soll darüber das Empfängersystem gesteuert und das Signal manipuliert werden können.



%\begin{description}
%	\item \textbf{Welche Fragen soll die Arbeit beantworten?}
%	\item \textbf{Welchem Zweck dient die Arbeit?}
%\end{description}


\section{Abgrenzung}


\section{Vorgehensweise}
Zu Beginn werden in Abschnitt 2 die theoretischen Grundlagen geschaffen, die für das Verständnis dieser Projektarbeit nötig sind. \\
Darauf folgt in Kapitel 3 die Anforderungsanalyse, sowie Planung und Aufbau des Empfängersystems mit Funkantenne.\\
In Kapitel 4 wird mit Hilfe von GNU Radio eine grafische Oberfläche gestaltet, über die sich beliebige Signale im gegebenen Frequenzbereich analysieren lassen.\\
Abschließend folgt eine Zusammenfassung mit Fazit und Ausblick.

%\begin{description}
%	\item \textbf{Welche Vorgehensweise wird hierzu gewählt?}
%\end{description}









