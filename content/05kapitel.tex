%!TEX root = ../dokumentation.tex

\chapter{Fazit}
Software Defined Radio Systeme haben in den letzten Jahren immer mehr an Bedeutung gewonnen, nicht zuletzt wegen den vergleichsweise geringen Kosten im Gegensatz zu klassischen Funkverarbeitungssystemen.
Bereits mit einem DVB-T-Stick und einer günstigen Teleskopantenne lassen sich je nach Anforderung brauchbare Ergebnisse erzielen. Bei der Analyse eines größeren Frequenzbereiches mit unterschiedlichsten Signalen und Anwendungen ist es jedoch unverzichtbar auf eine breitbandige, hochwertige Antenne samt SDR-Gerät mit ausreichender Bandbreite und Abtastrate zurückzugreifen. Die genauen Anforderungen hängen jedoch immer von der jeweiligen Anwendung ab und so ist es schwer, die Anforderungen an ein allgemeines Analysesystem zu formulieren bzw. zu erfüllen.

%Hilfreiche Fragestellungen
%\begin{description}
%\item Was sind die wichtigsten Ergebnisse?
%\item Was kann evtl. nicht beantwortet werden?
%\item Welchen Zweck haben die Ergebnisse (sowohl für die Arbeit an sich als
%auch für die Praxis / Wissenschaft)?
%\end{description}

%Ergebnis der wissenschaftlichen Arbeit
%Nach dem Hauptteil schreiben!
%Rückschluss von den Ergebnissen auf die Einleitung (Rückblick zur Einleitung herstellen)
%Keine Wiederholung aller Inhalte! Nur wichtigste Aspekte aufgreifen
%Wertvoll, wenn Hauptteil um neue Gedankengänge / Aspekte erweitert wird
%Unerwartete Ergebnisse bedeuten nicht, dass die Thesis gescheitert ist; sie müssen kritisch diskutiert werden

\section{Zusammenfassung}



\section{Ausblick}
Die Dezimeterwelle bietet viele praktische Anwendungsszenarien, die in deren rechtlichen Rahmen näher untersucht oder implementiert werden können:
Beispielsweise das Dekodieren von Wettersatellitenaufnahmen, die im Frequenzband um 1700 MHz gesendet werden. 